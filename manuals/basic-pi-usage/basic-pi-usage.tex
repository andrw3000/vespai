\documentclass[12pt, a4paper, oneside]{article}

\usepackage{geometry}

% Hyperlinks
\usepackage[dvipsnames]{xcolor}
\usepackage{hyperref}
\hypersetup{colorlinks=true, citecolor=RoyalBlue, linkcolor=PineGreen, linktoc=page, urlcolor=PineGreen}


\title{Basic usage and installation from disk image for VespAI on Raspberry Pi 4}
\author{Andrew Corbett}
\date{26$^{\mathrm{st}}$ August 2022}

\begin{document}
\maketitle

\section{Requirements}

\begin{itemize}
\item
Raspberry Pi 4 with at least 2GB of RAM.

\item
Micro SD card with at least 16GB of storage.

\item
SD card reader and a separate computer to flash the OS to the SD card.
\end{itemize}


\section{Flash SD Card}

With an SD card in you computer, you can wipe it and `flash' a new disk image to it, thus creating a carbon copy of the VespAI operating system for raspberry pi. One piece of software which executes this for you is Etcher: \url{https://www.balena.io/etcher/}.

\section{Connect remotely via Secure Shell}

\begin{enumerate}

\item
Using a keyboard, mouse and monitor, turn on Raspberry Pi, log in, and connect to local WiFi.

\item
Open terminal and execute \texttt{ip a} to print IP status. In the printout, under \texttt{eth0} look for the number next to \texttt{inet} excluding the slash and what follows. For example, the output of the \texttt{ip a} command may contain
\begin{verbatim}
	2: eth0: ...
		inet: 192.168.1.225/10 brd...
\end{verbatim}
In this case the IP address to connect to the server is \texttt{192.168.1.225}.

\item Shut down the Pi with \texttt{sudo halt} and disconnect from the power. You are now ready to reattach the Raspberry Pi in the field within range of this WiFi network. Plug into a power pack to turn on.

\item
On a remote device, connect to the server on the Pi via the \texttt{ssh} (Secure Shell) command. On macOS or linux this is completed via
\begin{verbatim}
	$ ssh detector@192.168.1.225
\end{verbatim}
followed by typing \texttt{yes} and enter (on the first connection). The password for \texttt{detector} is \texttt{alert}.

\item
On Windows one can download PuTTY; \url{https://www.putty.org}. Open this application and enter the IP address in the relevant box and click `OK'. Log in with
\begin{verbatim}
User name: detector
Password: alert
\end{verbatim}


\end{enumerate}

\section{Run experiment}

Now accessing the server by the remote device, move to the repo directory:
\begin{verbatim}
	$ cd vespai
\end{verbatim}
and run an experiment

\begin{verbatim}
	$ sudo python3 monitor/monitor_run.py -c=0.8 -s -p \
	-sd=monitor/detections/exp_name
\end{verbatim}
The slash is omitted and indicates a continuous line here. The flags give the option \texttt{-c} for the confidence threshold, \texttt{-s} for saving outputs, \texttt{-p} for printing on screen reports, \texttt{-sd} sets the save directory--change \texttt{exp\_name} to a specific experiment name.

\section{Export results}

Use the \texttt{scp} command (installed automatically with PuTTY) in terminal to save images onto the remote device:
\begin{verbatim}
	$ scp -r detector@192.168.1.225:/home/detector/vespai/monitor/detections \
	<choose/a/local/file/path>
\end{verbatim}
Change the IP address to that of the server.

\end{document}