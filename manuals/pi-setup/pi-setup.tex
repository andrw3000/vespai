\documentclass[12pt, a4paper, oneside]{article}

\usepackage{geometry}

% Hyperlinks
\usepackage[dvipsnames]{xcolor}
\usepackage{hyperref}
\hypersetup{colorlinks=true, citecolor=RoyalBlue, linkcolor=PineGreen, linktoc=page, urlcolor=PineGreen}


\title{Manual for setting up VespAlert on Raspberry Pi}
\author{Andrew Corbett}
\date{27$^{\mathrm{th}}$ May 2022}

\begin{document}
\maketitle

\section{Requirements}

\begin{itemize}
\item
Raspberry Pi 4 with at least 2GB of RAM.

\item
Micro SD card with at least 16GB of storage.

\item
SD card reader and a separate computer to flash the OS to the SD card.
\end{itemize}


\section{OS Installation}

\begin{enumerate}

\item
Download the Raspberry Pi Imager from \href{https://www.raspberrypi.com/software/}{\texttt{www.raspberrypi.com/software}}.

\item
Flash the 64 bit installation of Ubuntu to the SD card.

\item
With SD card in Pi, turn on and set up.

\end{enumerate}


\section{PyTorch Installation}

Open terminal and check pip3 is up to date:
\begin{verbatim}
	$ sudo apt-get install python3-pip 
\end{verbatim}
Install numpy:
\begin{verbatim}
$ sudo pip3 install numpy
\end{verbatim}
Install a nightly build of PyTorch and Torchvision for compatibility with the Pi's aarch64 (ARM) CPU. Check compatibility of these two packages for conflict: \url{https://github.com/pytorch/vision#instillation}.
\begin{verbatim}
$ sudo pip3 install --pre torch torchvision \
-f https://download.pytorch.org/whl/nightly/cpu/torch_nightly.html
\end{verbatim}
Amend the nightly release for YOLOv5 compatibility:
\begin{verbatim}
$ sudo nano /usr/local/lib/python3.10/...
dist-packages/torch/nn/modules/activation.py
\end{verbatim}
\begin{enumerate}
\item
Search with CTRL+W and type \texttt{return F.hardswish}
\item
Delete \texttt{self.inplace} from the arguments. (Exit and save with CTRL+X.)
\end{enumerate}
Install complimentary image libraries as not included in \texttt{torchvision>=10.0}:
\begin{verbatim}
$ sudo apt-get install libjpeg-dev
$ sudo apt-get install zlib1g-dev
$ sudo apt-get install libpng-dev
\end{verbatim}
To check package info:
\begin{verbatim}
$ pip3 show [package name]
\end{verbatim}
Add pillow to path:
\begin{verbatim}
export PATH="/home/detector/.local/bin:$PATH"
export PATH="/home/detector/.local/lib:$PATH"
\end{verbatim}
Amend depreciation in \texttt{pillow=9.0} of version number:
\begin{verbatim}
$ sudo nano /home/andy/.local/lib/python3.10/...
$ site-packages/PIL/__init__.py
\end{verbatim}
Declare
\begin{verbatim}
PILLOW_VERSION = __version__
\end{verbatim}
Install requirements file for YOLOv5:
\begin{verbatim}
$ sudo apt install -r path/to/yolov5/requirements.txt
\end{verbatim}


\section{For motion detection}
Install Open CV and imutils

\begin{verbatim}
$ sudo apt install python3-opencv
$ sudo pip install imutils
\end{verbatim}


\subsection{Fix for OpenCV function \texttt{cv2.imshow()}}

Responding to
\begin{verbatim}
Warning: Ignoring XDG_SESSION_TYPE=wayland on Gnome.
Use QT_QPA_PLATFORM=wayland to run on Wayland anyway
\end{verbatim}

\begin{enumerate}

\item
Disable Wayland by uncommenting \texttt{WaylandEnable=false} in
\begin{verbatim}
$ sudo nano /etc/gdm3/custom.conf
\end{verbatim}

\item
Add \texttt{QT\_QPA\_PLATFORM=xcb} in
\begin{verbatim}
$ sudo nano /etc/environment
\end{verbatim}

\item
Check whether you are on Wayland or Xorg using:
\begin{verbatim}
$ echo $XDG_SESSION_TYPE
\end{verbatim}
\end{enumerate}


\section{For YOLOv5}

\begin{verbatim}
$ pip install -qr https://raw.githubusercontent.com/
					ultralytics/yolov5/master/requirements.txt
\end{verbatim}

\end{document}