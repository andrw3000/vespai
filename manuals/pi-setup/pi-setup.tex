\documentclass[12pt, a4paper, oneside]{article}

\usepackage{geometry}

% Hyperlinks
\usepackage[dvipsnames]{xcolor}
\usepackage{hyperref}
\hypersetup{colorlinks=true, citecolor=RoyalBlue, linkcolor=PineGreen, linktoc=page, urlcolor=PineGreen}


\title{Manual for setting up VespAlert on Raspberry Pi}
\author{Andrew Corbett}
\date{21$^{\mathrm{st}}$ July 2022}

\begin{document}
\maketitle

\section{Requirements}

\begin{itemize}
\item
Raspberry Pi 4 with at least 2GB of RAM.

\item
Micro SD card with at least 16GB of storage.

\item
SD card reader and a separate computer to flash the OS to the SD card.
\end{itemize}


\section{OS Installation}

\begin{enumerate}

\item
Download the Raspberry Pi Imager from \href{https://www.raspberrypi.com/software/}{\texttt{www.raspberrypi.com/software}}.

\item
Flash the 64 bit installation of Ubuntu onto the SD card.

\item
With SD card in Pi, turn on and set up. Username: \texttt{detector}; computer name: \texttt{vespai}; pw: \texttt{alert}.

\item
Upgrade software
\begin{verbatim}
	$ sudo apt update
	$ sudo apt upgrade -y
\end{verbatim}
Then search for \texttt{Software Updater} and follow instructions.

\end{enumerate}


\section{Basic installation}
Open terminal and check software
\begin{verbatim}
	$ sudo apt-get update -y
\end{verbatim}

Install or update pip3:
\begin{verbatim}
	$ sudo apt-get install -y python3-pip 
\end{verbatim}

\section{VespAI software}
Save VespAI repo into home either by file transfer or cloning the GitHub repo
\begin{verbatim}
$ git clone https://github.com/andrw3000/vespai
\end{verbatim}


\section{PyTorch and dependencies}

Install complimentary image libraries as not included in \texttt{torchvision>=10.0}:
\begin{verbatim}
$ sudo apt-get install -y zlib1g-dev
$ sudo apt-get install -y libjpeg-dev
$ sudo apt-get install -y libpng-dev
\end{verbatim}
Install numpy:
\begin{verbatim}
	$ sudo pip3 install numpy
\end{verbatim}
Install a nightly build of PyTorch and Torchvision for compatibility with the Pi's aarch64 (ARM) CPU. This currently gives versions 1.12.0 and 0.13.0, respectively.
\begin{verbatim}
	$ sudo pip3 install --pre torch torchvision torchaudio \
	--extra-index-url https://download.pytorch.org/whl/nightly/cpu
\end{verbatim}
Amend installation for YOLOv5 compatibility as follows:
\begin{verbatim}
$ sudo nano /usr/local/lib/python3.10/...
dist-packages/torch/nn/modules/activation.py
\end{verbatim}
\begin{enumerate}
\item
Search with CTRL+W and type \texttt{return F.hardswish}
\item
Delete \texttt{self.inplace} from the arguments. (Exit and save with CTRL+X.)
\end{enumerate}
Perform test on importing torch and torchvision.
\begin{verbatim}
$ python3 -c "import torch, torchvision; \
print(torch.__version__, torchvision.__version__)"
\end{verbatim}
The returned output should read
\begin{verbatim}
UserWarning: Failed to load image Python extension"
\end{verbatim}
We can live with that. This manual was written with
\begin{verbatim}
torch.__version__=1.12.0
torchvision.__version__=0/13/0
\end{verbatim}
Check compatibility of these two packages at: \url{https://github.com/pytorch/vision#instillation}.
To check package info:
\begin{verbatim}
$ pip3 show [package]
\end{verbatim}
This will respond differently if one used sudo to install (and one should).
%Add pillow to path:
%\begin{verbatim}
%export PATH="/home/detector/.local/bin:$PATH"
%export PATH="/home/detector/.local/lib:$PATH"
%\end{verbatim}

Install requirements file for YOLOv5:
\begin{verbatim}
$ sudo pip3 install -r \
	/home/detector/vespai/models/yolov5/requirements.txt
\end{verbatim}


\section{For motion detection}
Install Open CV and imutils

\begin{verbatim}
$ sudo apt install python3-opencv
$ sudo pip install imutils
\end{verbatim}


\subsection{Fix for OpenCV function \texttt{cv2.imshow()}}

Responding to
\begin{verbatim}
Warning: Ignoring XDG_SESSION_TYPE=wayland on Gnome.
Use QT_QPA_PLATFORM=wayland to run on Wayland anyway
\end{verbatim}

\begin{enumerate}

\item
Disable Wayland by uncommenting \texttt{WaylandEnable=false} in
\begin{verbatim}
$ sudo nano /etc/gdm3/custom.conf
\end{verbatim}

\item
Add \texttt{QT\_QPA\_PLATFORM=xcb} in
\begin{verbatim}
$ sudo nano /etc/environment
\end{verbatim}

\item
Check whether you are on Wayland or Xorg using:
\begin{verbatim}
$ echo $XDG_SESSION_TYPE
\end{verbatim}
\end{enumerate}


\section{For YOLOv5}

\begin{verbatim}
$ pip install -qr https://raw.githubusercontent.com/
					ultralytics/yolov5/master/requirements.txt
\end{verbatim}

\end{document}